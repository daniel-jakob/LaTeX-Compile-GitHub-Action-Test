\begin{multicols}{2}
    % multicol parameters
    % These lengths are set only within the two main columns
    %\setlength{\columnseprule}{0.25pt}
    \setlength{\premulticols}{1pt}
    \setlength{\postmulticols}{1pt}
    \setlength{\multicolsep}{3pt}
    \setlength{\columnsep}{3pt}
    
    
    
    \chapter{Mechanics}
        $v=u+at$    \\
        $s=\frac{v+u}{2}t$ \\
        $v^2=u^2+2as$      \\
        $s=ut+\frac{1}{2}at^2$  \\
        $p=mv$    \\
        $\sum F= \frac{dp}{dt}$  \\
        $\int_{t_1}^{t_2}Fdt=p_2-p_1$ - Impulse \\
        $P=\frac{F}{A}$ \\
        $F=ma$\\
        $E_k=1/2mv^2$ \\
        $\mu=Fd_\perp$ - moment \\
        $E_p=mgh$\\
        $W=\int F \cdot dx=F\Delta x$\\
        $P_{avg}=\frac{\Delta W}{\Delta t}$\\
        $P_{inst}=\frac{dW}{dT}=F\cdot v$\\

        \chapter{Torque}
        $\tau=Fd_{\perp}$ - About midpoint\\
        $\tau=r \times F$ - vector torque\\
        $W=\tau(\theta_2-\theta_1)=\tau \Delta \theta$\\
        $P=\tau\omega$\\
        $L=r\times p=m r \times v $ - AM particle\\
        $L=I\omega$ -  AM rigid body\\
        $\sum \tau=\frac{dL}{dt}$\\

        \chapter{Rotational motion}
        $\omega=\frac{d\theta}{dt}$\\
        $\alpha=\frac{d\omega}{dt}=\frac{d^2\theta}{dt^2}$\\
        Equations for constant $\alpha$:\\
        $\theta=\theta_o+\omega_ot=\frac{1}{2}\alpha t^2$\\
        $\omega=\omega_o+\alpha t$\\
        $\omega^2=\omega_o^2+2\alpha(\theta-\theta_o)$\\
        $E_k=\frac{1}{2}I\omega^2$\\
        $I=\int r^2 dm$\\

        \chapter{SHM}
        $\omega=2\pi f=\frac{2\pi}{T}$\\
        $F=-kx$\\
        $E_s=\frac{1}{2}kx^2$\\
        $E=\frac{1}{2}mv^2+\frac{1}{2}kx^2=const$\\
        $T_s=2\pi \sqrt{\frac{m}{k}}$\\
        $T_{sp}=2\pi \sqrt{\frac{L}{g}}$\\
        $T_{physP}=2\pi \sqrt{\frac{I}{mgd}}$\\
    
        \chapter{Waves}
        $v=f\lambda$\\
        $k=\frac{2\pi}{\lambda}$\\
        $\omega=2\pi f$\\
        General Wavefunction for free wave:\\
        $y(x,t)=Asin(\omega t-kx)$\\
        Wave Equation:\\
        $\frac{\partial^2y}{\partial x^2}=\frac{1}{v^2}\frac{\partial^2y}{\partial t^2}$\\
        $E=hf=\hbar\omega=\frac{hc}{\lambda}$\\
        $eV_o=hf-\phi$ - photoelectric effect\\
        $f_L=\frac{v\pm v_L}{v \pm v_s}f_s$\\
        $c=1/\sqrt{\mu_o \varepsilon_o}$\\
        $n=c/v$\\
        $n_asin(\theta_a)=n_bsin(\theta_b)$\\
        $sin(\theta_c)=\frac{n_b}{n_a}$\\
        $dsin(\theta)=m\lambda$ - constructive\\
        $dsin(\theta)=(m+\frac{1}{2})\lambda$ - destructive\\
        $\frac{1}{u}+\frac{1}{v}=\frac{1}{f}$ - object,image distance\\

    \chapter{Elasticity}
    \newlength{\MyLen}
    \settowidth{\MyLen}{\texttt{letterpaper}/\texttt{a4paper} \ }
    $ Stress=F/A$\\
    $Strain=\delta l / l_o$\\
    \textit{Youngs Mod}$=Stress/Strain$\\

    \chapter{Thermodynamics}
    \settowidth{\MyLen}{\texttt{multicol} }
    $pV=nRT$\\
    $moles=m/A$\\
    Kinetic energy per molecule\\
    $E_k=n/2k_bT$- n=Degrees of freedom\\
    $v_{rms}=\sqrt{3k_bT/m}=\sqrt{3RT/A}$\\
    $\gamma=C_p/C_v$\\
    $\gamma=5/3$ - Monatomic\\
    $\gamma=7/5$ - Diatomic \\
    $v=\sqrt{\frac{\gamma P}{\rho}}$\\

    \chapter{Entropy-Heat}
    $dS=\frac{dQ}{T}$\\
    $Q=mc\Delta T$\\
    $Q=mL$\\
    $dQ=L dm$ - if $m$ changes\\
    Consider what changes, 
    i.e sign on \textit{Q}.
    Also for a large bath,
    $\Delta S=\frac{\Delta Q}{T}$\\
    $\frac{dQ}{dt}=k \frac{\Delta T}{ x}$\\

    \chapter{Relativity}
    $\beta=v/c$\\
    $\gamma=1/\sqrt{1-\beta^2}$\\
    $p=\gamma mv$\\
    $E=\gamma mc^2$\\
    $E^2=(pc)^2+(mc^2)^2$\\
    $\Delta t=\gamma \Delta t_o$\\
    $\Delta L= \frac{\Delta L_o}{\gamma}$\\
    $E_k=(1-\gamma)mc^2$\\

    \chapter{Electromagnetism}
    $F=\frac{Q_1 Q_2}{4\pi \varepsilon_o r^2} $\\
    $E=\frac{Q}{4\pi \varepsilon_o r^2}\hat{r}$\\
    $p=q\cdot l$ - Dipole moment\\
    $\tau=p \times E$ - Torque\\
    $u=-p\cdot E $ - Potential Energy\\
    $E=-\nabla V$\\
    $V=-\int E \cdot dl $\\
    $\int E \cdot dA=\frac{Q_{encl}}{\varepsilon_o}=\frac{\sum_i q_i}{\varepsilon_o}$\\
    $C=Q/V$\\
    $C=\varepsilon_o \frac{A}{d} $ - parallel plate\\
    $\frac{1}{C}=\frac{1}{C_1}+\frac{1}{C_2}+..+\frac{1}{C_n}$ - Series\\
    $C=C_1+C_2+..+C_n$ - parallel\\
    $U=\frac{Q^2}{2C}=\frac{1}{2}CV^2=\frac{1}{2}QV$ - stored E\\
    $u_e=\frac{1}{2}\varepsilon_o \varepsilon_r E^2$ - $E_E$ density\\
    $u_b=\frac{1}{2}\frac{B^2}{\mu_o \mu_r}$ - $E_B$ density\\
    $J=\sum_i n_iq_iv_{d_i}$\\
    $\rho=\frac{E}{J}$ - resistivity\\
    $\sigma=\frac{1}{\rho}$ - conductivity\\
    $R=\frac{\rho L}{A}$\\
    $V=\varepsilon-Ir$ - across a Battery\\
    $P=VI=I^2R=\frac{V^2}{R}$\\
    $I=\frac{dQ}{dt}=nqAv_d$\\
    $R=R_1+R_2+..+R_n$ - series\\
    $\frac{1}{R}=\frac{1}{R_1}+\frac{1}{R_2}+..+\frac{1}{R_n}$ - parallel\\
    $F=qE$\\
    $F=q \ v\times B$\\
    $F=Il\times B$ \\
    $B=\frac{\mu_o I}{2\pi r}$ - current carrying wire\\
    $B=\frac{\mu_o I}{2r}$ - center of a loop\\
    Multiply by N for N loops.
    $B=\frac{\mu_o I}{2(x^2+r^2)^{3/2}}$ - $x$ from center of loop\\
    $\mu_e=\frac{e\hbar}{2m_e}$ \\
    $\mu_N=\frac{e\hbar}{2m_N}$\\
    $\varepsilon=-\frac{d\Phi_b}{dt}=-\frac{B\cdot dA}{dt}$\\
    $E_b=-\mu_{N / e} \cdot B$\\
    $\int B \cdot dl=\mu_o I_{encl}$\\

    \chapter{Capacitors}
    Charging:\\
    $Q=Q_o(1-e^{-t/RC})$\\
    $I=I_o e^{-t/RC}$\\
    Discharging:\\
    $Q=Q_oe^{-t/RC}$\\
    $I=I e^{-t/RC}$\\
    Where $RC=\tau$ is the time constant\\

    \chapter{Gravity}
    $F=G\frac{Mm}{r^2}$\\
    $g=G\frac{M}{r^2}$\\
    
    \chapter{Quantum mechanics}
    $\left(-\frac{\hbar}{2m}\nabla+V\right)\psi=E\psi $\\
    $\Delta p \Delta x \geq \frac{\hbar}{2}$\\
    $f(E)=\frac{1}{e^{(E-E_f)/k_bT}+1}$\\
    $\lambda_db=\frac{h}{p}$\\
    \end{multicols}